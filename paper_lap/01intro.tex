\section{Introduction}
Laplace's equation arises in a vast array of contexts (electrostatics, harmonic functions, low-frequency acoustics, percolation theory, homogenization theory, and the study field enhancements in vacuum insulators for example) and serves as a useful model problem for the study of general elliptic partial differential equations (PDEs). As such, effective numerical methods for quickly and robustly solving Laplace's equation with high accuracy are desirable. Approaches based on potential theory proceed by reducing PDEs to second-kind boundary integral equations (BIEs), where the solution to the boundary value problem is represented by layer potentials on the boundary of the domain. Once these boundary integral equations are discretized the resulting linear systems are better-conditioned than those obtained by directly discretizing the PDE. When the boundary of the domain is smooth there are numerous methods for solving BIEs quickly and accurately (see \cite{hao}, for example). 

Near corners, however, the solutions to both the partial differential equations and corresponding boundary integral equations may have singularities, preventing the application of many traditional methods. Fortunately, a number of approaches have been developed to obviate this difficulty. One class of methods proceeds by introducing many additional degrees of freedom in the vicinity of the corners. In order to prevent the resulting linear systems from becoming intractably large one can use a variety of methods for {\it compressing} the linear system, effectively eliminating the extra degrees of freedom added in the vicinity of the corners. Moreover, the corner refinement and compression can be done in tandem resulting in fast and accurate solvers for elliptic PDEs (see \cite{helsing}, \cite{helsing2},  \cite{ojala}, \cite{helsjcp} and \cite{helsinv}  for  one approach called recursive compressed preconditioning, and  \cite{gillman}, \cite{bremer},\cite{bremer2}, and \cite{bremer3}  for other compression-based methods for solving Laplace's equation). Unfortunately, this approach becomes considerably more expensive in three dimensions limiting its application in that context.

 Another class of methods is based on approximating the solution to the two-dimensional problem by rational functions \cite{gopal2019solving} with poles exponentially clustered near the corners. While this approach allows for fast evaluation of the solution near the boundary of the domain, current implementations are specialized to two-dimensions, and do not scale well  for large problems.
 
 Finally, a recent approach is based on leveraging explicit representations of the solutions to the BIEs in the vicinity of the corner as sums of fractional powers depending on the angle \cite{serkhacha}. Using these representations one can construct high-order discretizations which introduce relatively few extra degrees of freedom near the corners (i.e. an amount which is comparable to the number required for smooth portions of the boundary). This approach has been used to generate efficient discretizations for Dirichlet problems for Laplace's equation on polygonal domains\cite{hoskins2019numerical}. 
 
 In this paper we describe a method for solving Laplace's equation on polygonal domains with Neumann boundary conditions given only a discretization of a corresponding Dirichlet problem. Our approach is based on using the discretization of a suitable adjoint problem. In particular, we show that if the transpose of the discretization of a suitable Dirichlet BIE is used, then the resulting solution will be accurate in a ``weak sense''; namely, it can be used to compute inner products with smooth functions accurately, though it cannot be interpolated. We then show how this solution can be used to obtain accurate solutions to the Neumann problem arbitrarily close to a corner by solving a set of local subproblems in the vicinity of that corner.
 
 The paper is organized as follows. In~\cref{sec:mprelim} we review relevant mathematical results associated with Laplace's equation. ~\cref{sec:bvp} describes the reduction of boundary value problems to boundary integral equations via potential theory, and reviews the analytic behavior of solutions near a corner. In~\cref{sec:nprelim,sec:napp} we present our numerical algorithm and the associated analysis. Finally, in~\cref{sec:num} we illustrate its application with several numerical experiments.