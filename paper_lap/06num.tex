
\section{Numerical results}
\subsection{Accuracy}
In this section, we demonstrate accuracy of the proposed numerical method (both in the "weak" sense, and 
"strong sense" after sufficiently many re-solves) on the triangular domain shown below. 
The reference solution for each of the examples is computed using a discretization with
a graded mesh in the vicinity of the corners, where the smallest panel at the corner is $2^{-150}$ times
the length of the first macroscopic panel away from the corner (see~\cref{fig:dom}). 
\begin{rem}
$2^{-150}$ is significantly smaller than machine precision, however the matrix entries corresponding to the
corner interactions can be computed accurately by recentering the corner at the origin. 
\end{rem} 
\begin{rem}
It is well-known that a smallest panel size of $2^{-100}$ is sufficient to resolve the density corresponding
to solutions of the Neumann problem in order to obtain full machine precision in the solution anywhere in the volume.
However, it is not true the value of the density is accurate to machine precision at all the nodes. In fact the quality
of the density deteriorates as one approaches the corner. Thus, in order to obtain accurate point values
of the density to machine precision at all points which are $2^{-100}$ away from the corner, we use a smallest
panel size of $2^{-150}$. 
\end{rem}
In these examples, the solutions are computed dense linear solves.

The potential at target locations which are more than 1 panel length away is the inner product of the 
density with a smooth function. For a target location $\boldsymbol{y}$, we compute the solution via the formula,
\begin{equation}
u(\by) = \int_{\Gamma} G(\bx,\by) \sigma(\bx) dS_{\bx} \approx \sum_{i=1}^{N}  G(\gamma(s_{i}),\by) \sigma_{i} \sqrt{w_{i}}
\end{equation}
In figure~\cref{fig:far-field}, we compute the error in the solution at target
locations in the volume for the right hand side given by
\begin{equation}
f(\bx) = 
\end{equation}

Another example of weak quantity is the polarization tensor associated with a domain. This requires the solution of the
exterior problems with boundary data $f_{1} = \bnu_{1}$ or $f_{2} = \bnu_{2}$. Let $\sigma_{1}$ and $\sigma_{2}$ denote
the corresponding solutions. The polarization tensor can be expressed in terms of the solutions $\sigma_{1}$ and $\sigma_{2}$
as
\begin{equation}
P = \begin{bmatrix}
\int_{\Gamma} x_{1} \sigma_{1}(\bx) dS_{\bx} & \int_{\Gamma} x_{2} \sigma_{1}(\bx) dS_{\bx} \\
\int_{\Gamma} x_{1} \sigma_{2}(\bx) dS_{\bx} & \int_{\Gamma} x_{2} \sigma_{2}(\bx) dS_{\bx} 
\end{bmatrix}
\end{equation}
The polarization tensor as computed by the reference solution, and the error in computation using the adjoint method
are given by
\begin{equation}
P = \begin{bmatrix}
\end{bmatrix} \, , \quad \text{Error} = 
\begin{bmatrix} 
\end{bmatrix}
\end{equation}

In order to demonstrate the capability to obtain the true density at the corner panels, we  apply the procedure
discussed in~\cref{sec:resolve} iteratively, and compare the obtained density with the reference density after 20,40,60, and
80 iterations of resolves in the vicinity of one of the corners. The reference density and the errors are shown in ~\cref{fig:dens-error}. 

After re-solving the density, we also compute the solution at target locations on a tensor product polar grid, where the grid
is exponentially spaced in the radial direction and equispaced in the angular direction. For target locations
close to panels which are not at the corner, we use adaptive integration in order to resolve the near-singular behavior of
the kernel for accurate computation of the integrals. For target locations close to the corner panel, since we do not have
the capability to interpolate the density, we use the underlying smooth quadrature rules for computing their contribution. 
The reference solution and the errors are demonstrated in~\cref{fig:vol-plot}.
\subsection{Performance}
In this section, we demonstrate the performance of the solver of two large domains. We solve scattering problems in the
exterior of a magnetron like region, and the maple leaf. The right hand side is the potential due to a collection of charges in the exterior of the region, and we impose that 