\documentclass[12pt,times]{elsarticle}
\usepackage{amsmath,amsthm,amssymb,amscd,fancybox,epsfig,float,subfigure}
\usepackage[margin=1.2in]{geometry}
\usepackage{graphics}
\usepackage{epstopdf}
\usepackage{multirow}
\usepackage{hhline}
\usepackage[nocompress]{cite}
\newcommand{\bx}{\boldsymbol {x}}

\begin{document}
 \title{Significance and novelty of this paper}
 \maketitle
The solution to boundary integral equations corresponding to Laplace's equations
with polygonal domains
develop singularities near corners; this poses a challenge for designing
efficient numerical methods for their solution. 
If the boundary data on each edge of the polygon is smooth, then the solution
to the corresponding boundary equation has an expansion terms of certain analytically
available singular powers. 
Using the known behavior of he solution to the integral equation, efficient
discretizations have been constructed for the Dirichlet problem.
However, these methods do not directly extend to the solution of the Neumann problem
since the leading order behavior of solutions to the Neumann problem is $O(t^{\mu})$ 
for $\mu \in (-1/2,1/2)$ depending on the angle at the corner, while that of solutions
to the Dirichlet problem is $O(C + t^{\mu})$ for $\mu>1/2$. 

In this paper, we present a universal discretization for the solution of the Neumann problem
on polygonal domains based on an adjoint Dirichlet discretization.
We show that the solution obtained in this manner is accurate in a ``weak-sense'', i.e. 
inner-products of the solution with smooth functions can be computed accurately. 
Further, we show that the solution is also accurate point wise away from the corners. 
We also present an algorithm to obtain accurate solutions in the vicinity of the corner
by solving a series of local sub-problems in the vicinity of the corner. 
This enables accurate evaluation of the corresponding potential at target locations arbitrarily
close to the corner. We present the effectiveness of our method, both in terms of accuracy
and speed through several numerical examples.
\end{document}

